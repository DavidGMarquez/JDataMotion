\chapter{Introdución}
\hyphenation{In-te-re-se}

Na actual sociedade da información, onde a cantidade de datos que se manexan aumenta día a día de xeito exponencial, a minería de datos convértese nunha ferramenta fundamental para poder explotalos de maneira eficaz, co fin último de xerar coñecemento a partir dos mesmos.

Para visualizar estes datos unha das técnicas máis utilizadas son os diagramas de dispersión ou scatterplots. Estes permítennos analizar os datos e atopar con facilidade relacións entre as distintas variables, como a correlación entre elas, a distribución dos puntos no plano, a tendencia dos datos recollidos ou outras características que sería complicado extraer a partir dun simple listado, posiblemente desordenado, de vectores de datos. Non obstante, os diagramas de dispersión restrínxennos a unha perspectiva estática do problema. En moitos deses problemas imos encontrar datos cunha compoñente que os sitúa no tempo. Con este proxecto pretendemos dotar a esta representación da súa perspectiva dinámica, para amosar os datos engadindo outro punto de vista que enriqueza a información extraida.

Deséxase desenvolver unha ferramenta para etiquetar cada punto dun diagrama de dispersión cun valor de significado temporal, de tal xeito que este puidese ser empregado como índice nunha visualización dinámica. Este valor numérico podería referenciar dende o momento de captación da tupla que a contén, ata unha ordenación dos datos atendendo á súa prioridade ou relevancia.

\section{Obxectivos xerais}

A motivación principal deste proxecto é o desenvolvemento dunha ferramenta capaz de visualizar a evolución dun conxunto de datos ao longo dunha magnitude como sería o tempo, ademais de permitir o preprocesado ou manipulación deses datos. Sendo máis específicos, este proxecto busca a realización da análise, deseño e implementación dunha aplicación que consiga: 

\begin{itemize}
\item Facilitarlle ao usuario o procesado de volumes de datos dun tamaño significativo. 
\item Posibilitar o traballo con formatos de arquivo CSV ou ARFF. 
\item Dispor das funcionalidades necesarias para manipular os datos. 
\item Ser capaz de amosar os datos en forma de diagramas de dispersión, con funcións de reprodución básicas. Tamén se debe posibilitar a configuración desta reprodución por parte do usuario. 
\item Aplicar filtros nos datos cos que se traballa, de xeito que se poidan eliminar datos fora dun rango, normalizar os seus valores, etc.
\item Interaccionar co usuario por medio dunha interface simple e amigable.
\item Aplicar nun caso real a ferramenta JDataMotion, para apreciar a súa utilidade.
\item Finalizar o desenvolvemento do proxecto antes do día 13 de Febreiro de 2015.
\end{itemize} 

\section{Relación da documentación}

Esta memoria plasma o proceso de desenvolvemento do proxecto JDataMotion, que persegue os obxectivos citados no apartado anterior.

Os distintos capítulos repártense do modo que segue:

\begin{description}
\item[Capítulo 1. Introdución:] composta por obxectivos xerais, relación da documentación que conforma a memoria, descrición do sistema (métodos, técnicas ou arquitecturas utilizadas e xustificación da súa elección).
\item[Capítulo 2. Planificación e presupostos:] inclúe a estimación dos recursos necesarios para desenvolver este proxecto, xunto co custo (presuposto) e planificación temporal do mesmo, así como a súa división en fases e tarefas.
\item[Capítulo 3. Especificación de requisitos:] inclúe a especificación do sistema, xunto coa información que este debe almacenar e as interfaces con outros sistemas, sexan hardware ou software, e outros requisitos (rendemento, seguridade, etc).
\item[Capítulo 4. Deseño:] rexistra como se realiza o sistema, a división deste en diferentes compoñentes e a comunicación entre eles. Así mesmo, neste apartado determínase o equipamento hardware e software necesario.
\item[Capítulo 5. Exemplos:] Avaliación do grao de cumprimento dos requisitos e tests os verifican.
\item[Capítulo 6. Conclusións e posibles ampliacións.]
\item[Apéndice A. Manuais técnicos:] incluirase toda a información precisa para aquelas persoas que se vaian a encargar do desenvolvemento e/ou modificación do sistema.
\item[Apéndice B. Manuais de usuario:] incluirán toda a información precisa para aquelas persoas que utilicen o sistema: instalación, utilización, configuración, mensaxes de erro, etc.
\item[Apéndice C. Licenza.]
\item[Bibliografía]
\end{description} 