\chapter{Análise de requisitos}

A extracción dos requisitos dun proxecto é unha fase fundamental na realización de calquera proxecto, pois inflúe non só nas propias tarefas a desenvolver para a súa implementación, se non tamén na valoración do produto final e da súa calidade. O proceso desta etapa pódese revisar na norma IEEE-STD-830-1998 \cite{ieee-std-830-1998}. A obtención de requisitos adóitase facer durante ou tras unha reunión cos clientes.

Durante a reunión cos clientes foron xurdindo requisitos ou condicións necesarias para o produto. Estes requisitos foron rexistrados para posteriormente seren ordenados e clasificados segundo certos criterios que se amosan a continuación.

Os casos de uso empréganse para modelar e representar cómo se vai realizar a interacción entre o sistema e os usuarios del, tamén coñecidos como actores. Os casos de uso constitúen as posibilidades das que dispón cada actor. Esta análise resulta especialmente útil en entornas orientadas a usuarios con distinta prioridade (un administrador, un usuario invitado, un usuario rexistrado, un usuario prémium, etc.) nas que cada un deses actores ten acceso a uns casos de uso específicos (por exemplo, moitas aplicacións web). Por tanto, a riqueza dos diagramas de casos de uso radica na variedade de tipos de usuario (actores). A nosa aplicación non necesita facer distinción algunha entre os tipos de usuario que poden facer uso dela. Todos van dispor das mesmas funcionalidades. Un escenario de caso de uso podería comezar cun traballador do ámbito médico-sanitario que dispón nun csv (exportado por outra aplicación, por exemplo) de certas medicións relacionadas cun paciente seu ao longo do seu seguimento. O traballador (usuario do sistema) podería importar o arquivo dentro do JDataMotion e obter unha táboa editable cos datos contidos, aplicar filtros para eliminar os datos atípicos das medicións do paciente e normalizar algunhas variables. Cos datos xa preprocesados, poderá reproducilos para ver o seu comportamento ao longo do tempo e a evolución do paciente.

\section{Requisitos funcionais}

\subsection*{RF01}
\begin{description}
\item[Título] \hfill \\
Importar arquivos con datos para o experimento
\item[Descrición] \hfill \\
A aplicación debe permitir cargar do sistema de arquivos un ficheiro que conteña unha secuencia de datos (nun formato axeitado segundo o RNF01) para ser utilizados no experimento.
\item[Importancia] \hfill \\
Esencial
\end{description}

\subsection*{RF02}
\begin{description}
\item[Título] \hfill \\
Exportar datos
\item[Descrición] \hfill \\
A aplicación debe permitir almacenar nun arquivo o conxunto de datos do experimento actual (tendo en conta filtrados, modificacións, datos engadidos ou eliminados...). Os arquivos de saída deberán respectar o RNF01 en canto a formato de almacenamento.
\subsubsection{Importancia}
Esencial
\end{description}

\subsection*{RF03}
\begin{description}
\item[Título] \hfill \\
Gardar sesión
\item[Descrición] \hfill \\
A aplicación debe permitir gardar en disco a sesión (ou experimento) actual tal e como está no momento de executar esta acción.
\item[Importancia] \hfill \\
Esencial
\end{description}

\subsection*{RF04}
\begin{description}
\item[Título] \hfill \\
Abrir sesión
\item[Descrición] \hfill \\
A aplicación debe permitir restaurar unha sesión (ou experimento) gardada anteriormente, de xeito que se atope exactamente igual ca no momento en que se gardou.
\item[Importancia] \hfill \\
Esencial
\end{description}

\subsection*{RF05}
\begin{description}
\item[Título] \hfill \\
Representar os datos en forma de táboa
\item[Descrición] \hfill \\
A aplicación debe ser capaz de amosar os datos segundo unha táboa na que figuren cabeceiras, tipos, valores, etc.
\item[Importancia] \hfill \\
Esencial
\end{description}

\subsection*{RF06}
\begin{description}
\item[Título] \hfill \\
Insertar datos no experimento actual
\item[Descrición] \hfill \\
A aplicación debe permitir a inserción dinámica de datos no experimento actual.
\item[Importancia] \hfill \\
Esencial
\end{description}

\subsection*{RF07}
\begin{description}
\item[Título] \hfill \\
Modificar datos no experimento actual
\item[Descrición] \hfill \\
A aplicación debe permitir a modificación dinámica de datos no experimento actual.
\item[Importancia] \hfill \\
Esencial
\end{description}

\subsection*{RF08}
\begin{description}
\item[Título] \hfill \\
Eliminar datos no experimento actual
\item[Descrición] \hfill \\
A aplicación debe permitir a eliminación dinámica de datos no experimento actual.
\item[Importancia] \hfill \\
Esencial
\end{description}

\subsection*{RF09}
\begin{description}
\item[Título] \hfill \\
Asignar tipos aos atributos dun arquivo importado
\item[Descrición] \hfill \\
A aplicación debe permitir especificar os tipos de atributos presentes no arquivo importado. Por exemplo, os datos cuantitativos poderían ser enteiros ou reais, mentres que os cualitativos serían algo distinto (mesmamente strings).
\item[Importancia] \hfill \\
Esencial
\end{description}

\subsection*{RF10}
\begin{description}
\item[Título] \hfill \\
Sinalar identificación temporal
\item[Descrición] \hfill \\
A aplicación debe permitir sinalar unha columna que exprese o orde ou a temporalidade dunha tupla, ou ben definir esta columna manualmente.
\item[Importancia] \hfill \\
Esencial
\end{description}

\subsection*{RF11}
\begin{description}
\item[Título] \hfill \\
Representar os datos graficamente mediante diagrama de dispersión
\item[Descrición] \hfill \\
A aplicación debe ser capaz de representar graficamente (mediante diagrama de dispersión) o conxunto de parámetros de entrada. Concretamente, débense poder representar ata 3 parámetros por cada diagrama de dispersión (ordeadas, abscisas e cor e forma dos puntos). Todos os diagramas de dispersión estarán englobados dentro do ``menú de visualización'', que cumprirá co RNF06.
\item[Importancia] \hfill \\
Esencial
\end{description}

\subsection*{RF12}
\begin{description}
\item[Título] \hfill \\
Engadir diagramas de dispersión ao menú de visualización
\item[Descrición] \hfill \\
A aplicación debe permitir engadir dinámicamente novos diagramas de dispersión dentro do menú de visualización.
\item[Importancia] \hfill \\
Esencial
\end{description}

\subsection*{RF13}
\begin{description}
\item[Título] \hfill \\
Eliminar un diagrama de dispersión do menú de visualización
\item[Descrición] \hfill \\
A aplicación debe permitir eliminar un diagrama de dispersión do menú de visualización.
\item[Importancia] \hfill \\
Esencial
\end{description}

\subsection*{RF14}
\begin{description}
\item[Título] \hfill \\
Configurar diagramas de dispersión do menú de visualización
\item[Descrición] \hfill \\
A aplicación debe permitir especificar para os diagramas de dispersión do menú de visualización a súa configuración, respecto a que parámetros se representarán en cada un dos eixos ou si as cores e formas dos puntos se desexan usar para representar algún atributo nominal.
\item[Importancia] \hfill \\
Esencial
\end{description}

\subsection*{RF15}
\begin{description}
\item[Título] \hfill \\
Detallar punto seleccionado dentro do diagrama de dispersión
\item[Descrición] \hfill \\
Cada punto dos diagramas de dispersión pode ser seleccionado para ver nun apartado os seus detalles (todos os seus atributos).
\item[Importancia] \hfill \\
Esencial
\end{description}

\subsection*{RF16}
\begin{description}
\item[Título] \hfill \\
Resaltar punto en diagramas de dispersión
\item[Descrición] \hfill \\
Cada punto seleccionalo dentro dun diagrama de dispersión resaltarase tanto nel coma en todos os demais diagramas de dispersión (que plasmarán outras proxeccións do mesmo punto).
\item[Importancia] \hfill \\
Esencial
\end{description}

\subsection*{RF17}
\begin{description}
\item[Título] \hfill \\
Desprazar a ventá de visualización por arrastre de cada diagrama de dispersión
\item[Descrición] \hfill \\
Para cada diagrama de dispersión poderemos usar unha ferramenta ``man'' para desprazar a ventá polo diagrama de dispersión.
\item[Importancia] \hfill \\
Esencial
\end{description}

\subsection*{RF18}
\begin{description}
\item[Título] \hfill \\
Escalar a ventá de visualización de cada diagrama de dispersión
\item[Descrición] \hfill \\
Para cada diagrama de dispersión poderemos usar unha ferramenta de escalado da ventá para facer zoom no diagrama de dispersión.
\item[Importancia] \hfill \\
Esencial
\end{description}

\subsection*{RF19}
\begin{description}
\item[Título] \hfill \\
Escalar e reposicionar dinamicamente
\item[Descrición] \hfill \\
Para cada diagrama de dispersión permitirase que a ventá de visualización que o enfoca se adapte dinamicamente ao conxunto de datos representados (movéndose, afastándose e aproximándose para englobar todos os datos).
\item[Importancia] \hfill \\
Esencial
\end{description}

\subsection*{RF20}
\begin{description}
\item[Título] \hfill \\
Reproducir a secuencia de datos
\item[Descrición] \hfill \\
A aplicación debe de permitir que a visualización dos diagramas de dispersión poida basearse na variable temporal (ou de orde) para reproducir a secuencia de datos, amosando os datos de cada diagrama de dispersión baixo unha secuencia de vídeo. Nesta secuencia engadiríase á visualización en cada instante a tupla de atributos asociada a esa marca temporal. 
\item[Importancia] \hfill \\
Esencial
\end{description}

\subsection*{RF21}
\begin{description}
\item[Título] \hfill \\
Representar estela
\item[Descrición] \hfill \\
A aplicación debe de permitir que cada novo punto pintado se ligue ao último representado no diagrama de dispersión por medio dunha liña recta.
\item[Importancia] \hfill \\
Esencial
\end{description}

\subsection*{RF22}
\begin{description}
\item[Título] \hfill \\
Difuminar estela ao longo da reprodución
\item[Descrición] \hfill \\
A aplicación debe permitir difuminar as estelas xa representadas a través do avance temporal.
\item[Importancia] \hfill \\
Esencial
\end{description}

\subsection*{RF23}
\begin{description}
\item[Título] \hfill \\
Configurar a reprodución da secuencia de datos
\item[Descrición] \hfill \\
A aplicación debe de permitir que a visualización dos diagramas de dispersión sexa configurable en canto a tempo transcorrido entre marcas temporais. Para a reprodución usando marcas temporais ponderadas, este tempo representará a separación entre as dúas marcas temporais mais próximas (tempo mínimo). Ademáis débese poder especificar o número de marcas temporais que durará o difuminado dos puntos que se ploteen, de xeito que durante ese intervalo cada punto se vaia difuminando ata desaparecer. Pode ser igual a 0 para que os puntos non se difuminen.
\item[Importancia] \hfill \\
Esencial
\end{description}

\subsection*{RF24}
\begin{description}
\item[Título] \hfill \\
Pausar a reprodución
\item[Descrición] \hfill \\
A aplicación debe permitir parar a reprodución na marca de tempo na que se atope ao executar esta acción, mantendo as visualizacións para ese momento.
\item[Importancia] \hfill \\
Esencial
\end{description}

\subsection*{RF25}
\begin{description}
\item[Título] \hfill \\
Ir a un determinado instante dentro do intervalo temporal da reprodución (GoTo)
\item[Descrición] \hfill \\
A aplicación debe permitir situarse directamente sobre un instante de tempo, mantendo a reprodución pausada sobre esa marca temporal, e visualizando os diagramas de dispersión tal e como deben estar nese momento.
\item[Importancia] \hfill \\
Esencial
\end{description}

\subsection*{RF26}
\begin{description}
\item[Título] \hfill \\
Insertar filtros para os datos do experimento
\item[Descrición] \hfill \\
A aplicación debe permitir engadir unha serie de filtros que se aplicarán de xeito secuencial sobre a secuencia de datos coa que se esté a traballar. Chamarémoslle ``secuencia de filtros'' a esta secuencia.
\item[Importancia] \hfill \\
Esencial
\end{description}

\subsection*{RF27}
\begin{description}
\item[Título] \hfill \\
Eliminar un filtro para os datos do experimento
\item[Descrición] \hfill \\
A aplicación debe permitir eliminar un determinado filtro dentro da secuencia de filtros.
\item[Importancia] \hfill \\
Esencial
\end{description}

\subsection*{RF28}
\begin{description}
\item[Título] \hfill \\
Configurar filtros para os datos do experimento
\item[Descrición] \hfill \\
A aplicación debe permitir seleccionar un determinado filtro dentro da secuencia de filtros para modificar a regla de filtrado implícita.
\item[Importancia] \hfill \\
Esencial
\end{description}

\subsection*{RF29}
\begin{description}
\item[Título] \hfill \\
Gardar unha secuencia de filtros do experimento
\item[Descrición] \hfill \\
A aplicación debe permitir gardar unha secuencia de filtros, non necesariamente correlativos, dentro dos que se estean aplicando sobre o experimento. Esta secuencia pode comprender tanto un só filtro como a secuencia de filtros enteira.
\item[Importancia] \hfill \\
Esencial
\end{description}

\subsection*{RF30}
\begin{description}
\item[Título] \hfill \\
Cargar unha secuencia de filtros para o experimento
\item[Descrición] \hfill \\
A aplicación debe permitir cargar do sistema de arquivos unha secuencia de filtros que se engadirá á cabeza da secuencia de filtros (a cal pode estar baleira). Esta secuencia tamén pode estar composta por un só filtro.
\item[Importancia] \hfill \\
Esencial
\end{description}

\subsection*{RF31}
\begin{description}
\item[Título] \hfill \\
Mover os filtros dentro da secuencia de filtros
\item[Descrición] \hfill \\
A aplicación debe permitir desprazar un filtro dentro da secuencia de filtros do experimento, de xeito que o orde de aplicación dos filtros varíe. O desprazamento realizarase inserindo o filtro en cuestión nunha nova posición.
\item[Importancia] \hfill \\
Esencial
\end{description}

\subsection*{RF32}
\begin{description}
\item[Título] \hfill \\
Configurar o menú de visualización
\item[Descrición] \hfill \\
A aplicación debe permitir cambiar os parámetros de visualización dos diagramas de dispersión que compoñen o menú de visualización, por exemplo, a cor das etiquetas e lendas, do fondo, dos eixos... ou a fonte, tamaño de letra...
\item[Importancia] \hfill \\
Optativa
\end{description}

\section{Requisitos de calidade}

\subsection*{RC01}
\begin{description}
\item[Título] \hfill \\
Latencia mínima para o procesamento
\item[Descrición] \hfill \\
A aplicación debe responder nun tempo razoable ás operacións executadas polo usuario, e intentar que esa latencia escale de xeito controlado ao aumentar a talla dos parámetros.
\item[Importancia] \hfill \\
Esencial
\end{description}

\section{Requisitos de deseño}

\subsection*{RD01}
\begin{description}
\item[Título] \hfill \\
Modularidade no deseño dos filtros
\item[Descrición] \hfill \\
A aplicación debe facilitar unha interface para a inclusión e uso de filtros personalizados por parte de calquera desenvolvedor de software que a implemente dentro do proxecto.
\item[Importancia] \hfill \\
Esencial
\end{description}

\section{Requisitos non funcionais}

\subsection*{RNF01}
\begin{description}
\item[Título] \hfill \\
Formatos de entrada admitidos ao importar e exportar arquivos
\item[Descrición] \hfill \\
A aplicación debe estar preparada para importar e exportar arquivos en distintos formatos, como son o CSV e ARFF.
\item[Importancia] \hfill \\
Esencial
\end{description}

\subsection*{RNF02}
\begin{description}
\item[Título] \hfill \\
Relación programa-sesión
\item[Descrición] \hfill \\
Cada instancia do programa debe traballar cunha única sesión (experimento).
\item[Importancia] \hfill \\
Esencial
\end{description}

\subsection*{RNF03}
\begin{description}
\item[Título] \hfill \\
Implementación en Java
\item[Descrición] \hfill \\
O software tense que desenvolver na linguaxe de programación Java.
\item[Importancia] \hfill \\
Esencial
\end{description}

\subsection*{RNF04}
\begin{description}
\item[Título] \hfill \\
Representación matricial dos diagramas de dispersión
\item[Descrición] \hfill \\
Os diagramas de dispersión represéntanse de xeito matricial, facendo que cada parámetro dentro dun eixo sexa enfrontado a cada un dos demais do outro eixo, e en cada punto desa dupla se sitúe o diagrama de dispersión que compara ambos parámetros. Deste xeito, os diagramas de dispersión non son acumulables: se temos un que representa X (abscisas) fronte a Y (ordenadas), non podemos engadir outro que represente X (abscisas) fronte a Y (ordenadas), pois ocuparían ambos a mesma cela dentro da matriz de diagramas de dispersión.
\item[Importancia] \hfill \\
Esencial
\end{description}

\subsection*{RNF05}
\begin{description}
\item[Título] \hfill \\
Entrega dentro de prazo
\item[Descrición] \hfill \\
Débese entregar unha versión funcional e documentada antes do día 13 de Febreiro de 2014, ás 14:00 horas, pois é o momento no que remata o prazo de entrega.
\item[Importancia] \hfill \\
Esencial
\end{description}

\section{RFs dos sprints}

Imos a detallar a asignación de requisitos funcionais (RFs) aos distintos sprints ao longo da fase de Desenvolvemento, asignando uns prazos aproximados de traballo sobre uns conxunto de RFs relacionados entre si.

\subsection*{Sprint 01}
\begin{description}
\item[Nome:] Interacción co sistema de ficheiros
\item[Fase:] Desenvolvemento
\item[Comezo:] 17/02/2014
\item[Finalización:] 24/02/2014
\item[RFs a implementar:] RF01, RF02, RF03, RF04
\end{description}

\subsection*{Sprint 02}
\begin{description}
\item[Nome:] Manipulación de datos
\item[Fase:] Desenvolvemento
\item[Comezo:] 24/02/2014
\item[Finalización:] 10/03/2014
\item[RFs a implementar:] RF05, RF06, RF07, RF08
\end{description}

\subsection*{Sprint 03}
\begin{description}
\item[Nome:] Preprocesado
\item[Fase:] Desenvolvemento
\item[Comezo:] 10/03/2014
\item[Finalización:] 24/03/2014
\item[RFs a implementar:] RF09, RF10
\end{description}

\subsection*{Sprint 04}
\begin{description}
\item[Nome:] Visualización dos datos
\item[Fase:] Desenvolvemento
\item[Comezo:] 24/03/2014
\item[Finalización:] 14/04/2014
\item[RFs a implementar:] RF11, RF12, RF13, RF14
\end{description}

\subsection*{Sprint 05}
\begin{description}
\item[Nome:] Ferramentas de visualización
\item[Fase:] Desenvolvemento
\item[Comezo:] 14/04/2014
\item[Finalización:] 28/04/2014
\item[RFs a implementar:] RF15, RF16, RF17, RF18, RF19
\end{description}

\subsection*{Sprint 06}
\begin{description}
\item[Nome:] Reprodución
\item[Fase:] Desenvolvemento
\item[Comezo:] 28/04/2014
\item[Finalización:] 05/05/2014
\item[RFs a implementar:] RF20
\end{description}

\subsection*{Sprint 07}
\begin{description}
\item[Nome:] Configuración da reprodución
\item[Fase:] Desenvolvemento
\item[Comezo:] 05/05/2014
\item[Finalización:] 19/05/2014
\item[RFs a implementar:] RF21, RF22, RF23
\end{description}

\subsection*{Sprint 08}
\begin{description}
\item[Nome:] Funcións de reprodución
\item[Fase:] Desenvolvemento
\item[Comezo:] 19/05/2014
\item[Finalización:] 02/06/2014
\item[RFs a implementar:] RF24, RF25
\end{description}

\subsection*{Sprint 09}
\begin{description}
\item[Nome:] Filtros
\item[Fase:] Desenvolvemento
\item[Comezo:] 02/06/2014
\item[Finalización:] 16/06/2014
\item[RFs a implementar:] RF26, RF27, RF28
\end{description}

\subsection*{Sprint 10}
\begin{description}
\item[Nome:] Xestionar filtros
\item[Fase:] Desenvolvemento
\item[Comezo:] 16/06/2014
\item[Finalización:] 30/06/2014
\item[RFs a implementar:] RF29, RF30, RF31
\end{description}

\subsection*{Sprint 11}
\begin{description}
\item[Nome:] Outras funcións de visualización
\item[Fase:] Desenvolvemento
\item[Comezo:] 30/06/2014
\item[Finalización:] 07/07/2014
\item[RFs a implementar:] RF32
\end{description}

A partir da planificación temporal de grupos de RFs en común podemos estimar con certa confianza a duración total do proxecto, polo menos na súa fase de desenvolvemento. Faltarían dúas fases máis por considerar:

\begin{itemize}
\item Fase de inicio, previa á execución dos sprints, duraría un prazo de 2 semanas e constaría das seguintes tarefas:
\begin{itemize}
\item Lectura da especificación e documentación
\item Reunión cos directores do proxecto
\item Redacción do anteproxecto
\end{itemize} 
\item Fase de documentación, posterior á execución dos sprints, duraría un prazo de 5 semanas e constaría das seguintes tarefas:
\begin{itemize}
\item Revisión e documentación do código
\item Compilación de documentos cos sprints
\item Redacción da memoria
\item Redacción dun manual de usuario
\item Reunión co director para revisión
\end{itemize}
\end{itemize} 

Agora que temos as estimacións da fase de inicio, da fase de documentación e da fase de desenvolvemento (cos seus sprints estimados) podemos realizar un diagrama de Gantt do proxecto (figura \ref{gantt}) para establecer a liña base do mesmo:

\begin{figure}
\centering
\includegraphics[width=\textwidth,height=\textheight,keepaspectratio]{figuras/gantt}
\caption{Diagrama de Gantt}
\label{gantt}
\end{figure}

Considerando que coñecemos as tarefas da fase de inicio e da fase de documentación, así como os sprints da fase de desenvolvemento e incluso os RFs a desenvolver dentro de cada un deles, podemos plasmar todas estas tarefas nun Esquema de Descomposición do Traballo ou EDT (figura \ref{edt}). Isto implica que os RFs (ou máis ben o seu desenvolvemento) van ter a consideración de tarefas a partir de agora no noso proxecto.

\begin{figure}
\centering
\includegraphics[width=\textwidth,height=\textheight,keepaspectratio]{figuras/edt}
\caption{Esquema de Descomposición do Traballo (EDT)}
\label{edt}
\end{figure}