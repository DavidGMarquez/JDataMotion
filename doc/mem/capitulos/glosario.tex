\pagestyle{plain}
\chapter*{Glosario}
\begin{description}
  \item[Sesión] \hfill \\
  Interacción co sistema e as súas funcionalidades. Inclúe a importación duns datos, o seu procesado, filtrado e visualización.
  \item[Experimento] \hfill \\
  Ver sesión.
	\item[Diagrama de dispersión] \hfill \\
  Diagrama matemático que fai uso das coordenadas cartesianas para amosar os valores de dúas variables para un conxunto de datos. Cada punto no diagrama referencia un valor ao longo do eixo de ordenadas e outro ao longo do eixo de abscisas.
	\item[Modelo] \hfill \\
  Estrutura e contido dos datos cos que se traballa no experimento. Abrangue tanto os tipos dos atributos coma os seus valores dentro de cada entrada, así coma o nome do conxunto de datos, ou a marca do atributo que actúa de índice temporal.
	\item[Filtro] \hfill \\
  Ferramenta configurable que en función dos seus parámetros pode converter, a partir dun modelo A de entrada, un modelo B de saída.
	\item[Visualización] \hfill \\
  Presentación gráfica da relación, neste caso baixo a forma de diagramas de dispersión.
	\item[Reprodución] \hfill \\
  Activación dunha visualización para que cambie o seu estado ao longo do tempo, neste caso en función dun índice ou variable temporal.
	\item[Índice temporal] \hfill \\
  Atributo do modelo que representa a orde na que os datos foron tomados, foron detectados, queren ser priorizados ou simplemente se desexan amosar. Se non se define, asúmese como índice temporal a orde das entradas do modelo.
	\item[Entrada] \hfill \\
  Ver instancia.
	\item[Instancia] \hfill \\
  Vector de datos que contén un valor para cada atributo do modelo. Pode conter valores nulos.
	\item[Atributo] \hfill \\
  Cada unha das variables coas que traballa o modelo.
	\item[Atributo numérico] \hfill \\
  Atributo que só pode tomar valores numéricos.
	\item[Atributo de tipo data] \hfill \\
  Atributo que só pode tomar valores de tipo data.
	\item[Atributo de tipo string] \hfill \\
  Atributo que pode tomar calquera valor en forma de cadea de caracteres.
	\item[Atributo nominal] \hfill \\
  Atributo que só pode tomar unha serie de valores concretos.
	\item[Relación] \hfill \\
 Ver modelo.
\item[Reprodución] \hfill \\
 Visualización dinámica da relación, é dicir, presentación dunha relación na que cada entrada se visualiza nun momento do tempo determinado.
\item[Comando] \hfill \\
 Entidade que representa unha orde para o sistema. Contén información sobre o artefacto ao que afecta, o tipo de orde que se expresa e os parámetros necesarios para a súa execución.
\end{description}