\chapter{Planificación e presupostos}

Planificación e presupostos: debe incluír a estimación do costo (presuposto) e dos 
recursos necesarios para efectuar a implantación do Traballo, xunto coa planificación 
temporal do mesmo e a división en fases e tarefas. Recoméndase diferenciar os costos relativos a persoal dos relativos a outros gastos como instalacións e equipos.

A estimación dos custos de desenrolo do proxecto amósase no cadro \ref{tab:custosLabel}. Para ela, consideramos a adquisición dun novo equipo informático. As horas de traballo neste caso non van ter un valor económico asociado, como consecuencia de que este proxecto pertenza a un Traballo de Fin de Grao, pois as horas do traballo do alumno correspóndense coas que este debe cumprimentar para a obtención do título. Para o consumo eléctrico tivemos en conta o prezo do kWh en España\cite{preciokWh_espana} e fixemos unha estimación\cite{energyusecalculator} do consumo en Watios dun equipo informático, que a plena potencia pode traballar a 120 vatios. Considerando que o desenrolo  do traballo durará 401.25 horas, necesitaremos 120 W * 401.25 h = 48150 Wh = 48,15 kWh. Ademais, tratarase de buscar solucións de código aberto ás distintas necesidades de módulos adicionais que vaian xurdindo, polo que o custo económico destas solucións vai ser nulo.

% Table generated by Excel2LaTeX from sheet 'Hoja1'
\begin{table}[htbp]
  \centering
    \begin{tabular}{rrrrr}
    \textbf{Activo} & \textbf{Cantidade} & \textbf{C.U. sen IVE} & \textbf{IVE} & \textbf{Custo total} \\
    Ordenador portátil & 1     & 570,00 \euro & 21\%  & 689,70 \euro \\
    Horas de traballo & 401,25 horas & 0,00 \euro/hora & 21\%  & 0,00 \euro \\
    Consumo eléctrico & 48,15 kWh & 0,12 \euro/kWh & 21\%  & 7,23 \euro \\
          &       &       & \textbf{Total} & 696,93 \euro \\
    \end{tabular}%
		\caption{Custos}
  \label{tab:custosLabel}%
\end{table}%

